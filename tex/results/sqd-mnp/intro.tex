\section{Semiconducting Quantum Dot-Metal Nanoparticle Hybrid}\label{sec:sqd-mnp}

In order to test the proposed PEOM method detailed in \cref{sec:peom}, we model a
semiconducting quantum dot (SQD) as the quantum \note{(primary)} system and a metal
nanoparticle (MNP) as the classical \note{(auxiliary)} system. The SQD-MNP
hybrid system has been studied extensively both theoretically (REFS) and
experimentally (REFS). In particular, in Ref.~\cite{Zhang2006}, Zhang et al.
introduce a simple theoretical model in which the SQD is treated as a 
two-level, atomic-like quantum system whose dynamics are solved using the
density matrix formalism of quantum mechanics. The MNP on the other hand is
treated classically and the systems' dynamics are coupled through the external
field. Indeed, this method provided the basis for the PEOM method.

For a monochromatic external field, the energy absorption rate of the hybrid
system can be calculated analytically by means of the rotating wave
approximation (RWA). In \cref{sec:en_abs_rates}, we compare examples of these
analytical solutions and show agreement with the PEOM method.

For a pulsed external field, population inversion in the SQD can be obtained
and modified by the presence of the MNP. For slowly-varying pulse envelopes,
the density matrix equations of motion for the SQD can be solved
semi-analytically within the RWA and in \cref{sec:pop_inv}, we show agreement
with the PEOM method for picosecond pulses. However, in the case of ultrafast,
femtosecond pulses, we show that the RWA breaks down and the PEOM method must
be preferred for more accurate results.
