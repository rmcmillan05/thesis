\chapter{Fourth-Order Runge-Kutta Method for Solving First-Order ODEs}\label{app:runge-kutta}

Suppose we have a first-order differential equation of the form
%
%
\begin{equation}
    \dot y = f(t,y) \ ,
\end{equation}
%
%
where $y\equiv y(t)$, and we want to approximate the solution over some interval $t\in [a,b]$ ($b>a$).
We begin by splitting the interval into $N+1$ equally-spaced points, $\{t_0,t_1,\ldots,t_N\}$ with $t_0=a$ and $t_N=b$, such that
%
%
\begin{equation}
    t_n = a + nh \ , \qquad n = 0,1,\ldots, N \ , \label{eq:runge t_k}
\end{equation}
%
%
where $h$ is the spacing, or step-size, given by
%
%
\begin{equation}
    h = \frac{b-a}{N} \ .
\end{equation}
%
Provided $y(t_0)=y(a)$ (the initial condition) is known, the fourth-order
Runge-Kutta method approximates $\{ y(t_1),y(t_2),\ldots y(t_N)\}$ using the
following algorithm:~\cite{burden}


\fbox{
\begin{minipage}{0.85\textwidth}
{\tt
    \noindent For $\mathtt{n = 0, 1, \ldots, N-1}$,
%
%
%\begin{subequations}\label{eq:runge-kutta alg}
\begin{align*}
    &\mathtt{k_1 = h\ f(t_n, y_n)\ ,} \\
    &\mathtt{k_2 = h\ f(t_n + \frac{1}{2} h, y_n + \frac{1}{2} k_1)\ ,}\\
    &\mathtt{k_3 = h\ f(t_n + \frac{1}{2} h, y_n + \frac{1}{2} k_2)\ ,}\\
    &\mathtt{k_4 = h\ f(t_n +  h, y_n + k_3) \ ;}\\
    &\mathtt{\qquad y_{n+1} = y_n + \frac{1}{6} (k_1+2k_2+2k_3+k_4) \ ;}\\
    &\mathtt{\qquad \quad \  t_{n+1} = t_n + h \ .}
\end{align*}
%\end{subequations}
%
%
where we have defined
%
%
\begin{equation*}
    \mathtt{y_n \equiv y(t_n) \ .}
\end{equation*}
}
\vspace{-1.5em}
\end{minipage}
}
