\subsection{Energy Absorption Rates}\label{sec:en_abs_rates}

The energy absorption rate (EAR) of the SQD-MNP system is a steady-state
property in response to the monochromatic wave given in \cref{eq:plane_wave}. Suppose a
steady-state is achieved (i.e. $\dot\Delta=\dot{\bar\rho}_{21}=0$) at
$t=T^\text{s.s}$. Then we define 
%
\begin{subequations}
\begin{align}
    \Deltass &= \Delta\left(T^\text{s.s}\right) \ ,\\
    \rhoss &= \bar\rho_{21}\left(T^\text{s.s}\right)  \ .
\end{align}
\label{eq:steady-state}
\end{subequations}
%
The EAR of the SQD is defined as 
%
\begin{equation}
    \qsqd = \frac{1}{2}\hbar\omega_0\Gamma_{11}\left( 1-\Deltass \right)\ ,
    \label{eq:qsqd}
\end{equation}
%
while that of the MNP is
%
\begin{equation}
    \qmnp = \left\langle \int \bm{j}\left(T^\text{s.s.}\right)\cdot
    \bm{\emnpin}\left(T^{s.s}\right)
    dV\right\rangle \ ,
%    \qmnp = \left\langle \bm{j}\cdot \bm{\emnpin} \right\rangle V \ ,
    \label{eq:qmnp}
\end{equation}
%
where
%
\begin{equation}
    \bm j(t) = \frac{1}{V}\frac{d}{d t} \pmnp(t) 
    \label{eq:current_density}
\end{equation}
%
is the current density in the MNP~\cite{artuso_strongly_2010}\note{(is this the
correct formula for current density?)} and
$\emnpin(t)$ is the field inside the MNP. For a sphere it can be shown
that~\cite{landau_electrodynamics_1984}
%
\begin{equation}
    \emnpin(t) = \emnp(t) - \frac{4}{3}\pi \pmnp(t) \ .
    \label{eq:emnpin}
\end{equation}
%
We define the time-average of a function $h(t)$ as
%
\begin{equation}
    \langle h(t)\rangle \equiv \frac{1}{\delta T}\int_{t}^{t+\delta T} h(t')dt'
    \ ,
    \label{eq:time-average}
\end{equation}
%
and choose $\delta_t$ to be one period of the wave,
%
\begin{equation}
    \delta t = \frac{2\pi}{\omega_L} \ .
    \label{eq:delta_t}
\end{equation}
%
\\

\noindent{\it {\large (i) Solution in the PEOM}}\\

In the PEOM solution (see
\cref{sec:sqd-peom}), $\Delta(t)$ is found by numerically solving
\cref{eq:eoms} and so $\Deltass$, and therefore $\qsqd$ (\cref{eq:qsqd}), can
be calculated trivially assuming the simulation time is long enough that a
steady-state can be said to have been reached.

To calculate $\qmnp$ from
\cref{eq:qmnp}, one must first calculate the current density from
\cref{eq:current_density}. The derivative, $\delta/\delta t\pmnp(t)$, may be
calculated numerically though it can be shown from
\cref{eq:pmnpt_peom,eq:s_k_mnp} (see Appendix REF) that
%
\begin{equation}
    \frac{d\pmnp}{d t} = \sum_{k=1}^N c_k \left(
    \omega_k\imag{s_k(t)}
        -\gamma_k\real{s_k(t)}\right) \ .
    \label{eq:dpdt}
\end{equation}
%
The time-average in \cref{eq:qmnp} can then be found by numerical
integration. For a small time-step, $h$, in the fourth order Runge-Kutta
method, we found the crude approximation,
%
\begin{equation}
    \int_{t_n}^{t_m} f(t')dt' \approx h \sum_{i=n}^{m} f(t_i) \ ,
    \label{eq:numerical_int}
\end{equation}
%
to be sufficient.\\

\noindent{\it {\large (ii) Analytical Solution}}\\

As mentioned in \cref{sec:sqd-anal}, $\Deltass$, and therefore $\qsqd$, can be
found analytically by solving \cref{eq:eoms} within the RWA.

Within the RWA, it can be shown that field felt by the MNP and its polarization
are given, respectively, by (see Appendix REF)
%
\begin{align}
    \emnp(t) &= \bemnp(t)
    e^{-i\omega_L t} + \text{c.c.} \ ,\label{eq:emnp_rwa} \\
    \pmnp(t) &= \epsb \alphamnp(\omega_L)\bemnp(t)
    e^{-i\omega_L t} + \text{c.c.} \ , \label{eq:pmnp_rwa}
    \qquad
\end{align}
%
where
%
\begin{equation}
    \bemnp(t) = \frac{E_0}{2} + \frac{g\mut}{\epsb R^3}\bar\rho_{21}(t) \ .
    \label{eq:emnp_bar}
\end{equation}

If we substitute \eqref{eq:emnp_rwa,eq:pmnp_rwa} into Eq.~\eqref{eq:qmnp} and
assume then using the RWA yields the following (see Appendix REF),
%However, $\frac{\partial\pmnp}{\partial t}\pmnp=\frac{1}{2}\frac{\partial}{\partial
%t}\left(\pmnp^2\right)$ and it can be shown that $\pmnp^2$ is independent of $t$ in the RWA which
%means $\frac{\partial}{\partial t}\left( \pmnp^2 \right)=0$. Hence, the second
%term in~\eqref{eq:qmnp_RWA} vanishes so that in the RWA we can replace
%$\emnpin$ with $\emnp$ in Eq.~\eqref{eq:qmnp} which can be calculated to give
%[check $\epsb$ factor]
%
%
\begin{equation}
    \qmnp = 2\omega_L \imag{\alphamnp(\omega_L)} \left|\frac{E_0}{2} +
    \frac{g\mut}{\epsb R^3}\rhoss \right|^2 \ .
%    &\left\{ 
%        \frac{E_0^2}{4}
%        + \frac{g\mut E_0}{\epsb R^3}\real{\bar\rho_{21}}
%        + \left(\frac{g\mut}{\epsb R^3}|\bar\rho_{21}|\right)^2
%        \right\}  . \
    \label{eq:qmnp_ss}
\end{equation}
%
%where
%%
%\begin{equation}
%    \bemnp = \frac{E_0}{2} + \frac{g\mut}{R^3}\rhoss \ .
%    \label{eq:emnp_bar_ss}
%\end{equation}
