\subsection{Analytical and Semi-Analytical Solutions}\label{sec:sqd-anal}

To solve \cref{eq:eoms} analytically, we first separate out the slowly and quickly oscillating terms of the
off-diagonal density matrix elements:
%
\begin{subequations}
    \begin{align}
        \rho_{12}(t) = \bar\rho_{12}(t)e^{\imagn\omega_L t} \ , \\
        \rho_{21}(t) = \bar\rho_{21}(t)e^{-\imagn\omega_L t} \ , 
    \end{align}
    \label{eq:rhobar}
\end{subequations}
%
where $\bar\rho_{12}(t)$ and $\bar\rho_{21}(t)$ are assumed to vary much larger
timescale than $2\pi/\omega_L$. We assume a {\it slowly-varying} pulse
envelope so that $f(t)$ also varies on a timescale larger that $2\pi/\omega_L$.
Thus, from \cref{eq:esqd}, we can express the field felt by the SQD
approximately as (see Appendix REF)
%
\begin{equation}
    \esqdeff(t) = \frac{\hbar}{\mut}\left[ \left( \frac{\Omega(t)}{2}e^{-\imagn\omega_L t} +
    G\rho_{21}(t) \right) + \text{c.c.} \right] ,
    \label{eq:effective_field}
\end{equation}
%
where
%
\begin{subequations}
    \begin{align}
        \Omega(t) &= f(t) \omegaeff \ , \label{eq:omega} \\
        \omegaeff &=  \Omega_0 \left[ 1 +
        \frac{g}{R^3}\alphamnp(\omega_L) \right] \ ,\label{eq:omegaeff}\\
        G &= \frac{g^2\mut^2 }{\hbar\epsb R^6}\alphamnp(\omega_L) \ , \label{eq:G}
    \end{align}
\end{subequations}
%
where
%
\begin{equation}
    \Omega_0 = \mut E_0/\hbar 
    \label{eq:omega_0}
\end{equation}
%
is the Rabi frequency of the isolated SQD.

One may then substitute \cref{eq:effective_field} into the density matrix EOMs
in \cref{eq:eoms} and determine a numerical solution (e.g. using the
fourth-order Runge-Kutta method). This is equivalent to the method used by Yang
et al. in REF. Alternatively, one can make use of the rotating wave
approximation (RWA) which assumes that the carrier frequency is close to
resonant with the exciton frequency ($\omega_L\approx\omega_0$) so that terms
oscillating at frequencies far from $\omega_0$ can be neglected. By
substituting \cref{eq:effective_field} into \cref{eq:eoms} and employing the RWA,
one arrives at the following, modified RWA EOMs for the density matrix (see
Appendix REF)
%
\begin{equation}
    \begin{cases}
        \dot\Delta &= 4\imag{\left( \frac{\Omega}{2}+G\bar\rho_{21} \right)\bar\rho_{12}}
        + \Gamma_{11}\left( 1-\Delta \right) \\
        \dot{\bar\rho}_{21} &= \left[ \imagn\left( \omega_L-\omega_0 +G\Delta
        \right)-\Gamma_{21} \right]\bar\rho_{21} + \imagn\frac{\Omega}{2}\Delta 
    \label{eq:eoms_rwa}
    \end{cases}
    \ .
\end{equation}
%
One can then numerically solve these modified EOMs to determine the dynamics of the
slowly-varying parts, $\bar\rho_{21}(t)$, which should be similar to the full
dynamics of $\rho_{21}(t)$ provided the RWA is valid. This is equivalent to the
method used in REFs.

In the case of a rectangular envelope of unitary height ($f(t)=1$), i.e. a
monochromatic wave, 
%
\begin{equation}
    \eext(t) = E_0\cos(\omega_L t) \ ,
    \label{eq:plane_wave}
\end{equation}
%
\cref{eq:eoms_rwa} may be solved analytically under the steady-state conditions
$\dot\Delta=\dot{\bar\rho}_{21}=0$ to obtain
%
\begin{equation}
    \rhoss =
    \frac{-\omegaeff\Deltass}{\omega_L-\omega_0+G\Deltass
        +\imagn\Gamma_{21}},
    \label{eq:steady-state}
\end{equation}
%
where $\Deltass$ is found by solving a cubic equation (see Appendix REF).

\cref{eq:steady-state} allows us to calculate the energy absorption rate of the
system, while we can investigate population inversion by solving the density
matrix EOMs directly.
